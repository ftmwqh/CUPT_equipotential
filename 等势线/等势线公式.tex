\documentclass{report}
\usepackage{ctex}
\usepackage{graphicx}
\usepackage{amsmath}
\usepackage{indentfirst}
\usepackage{titlesec}
\usepackage{setspace}
\usepackage{subfigure}
\usepackage{caption}
\usepackage{float}
\usepackage{booktabs}
\usepackage{geometry}
\usepackage{multirow}
\usepackage{color}
\geometry{left=4cm,right=4cm,top=3cm}
\title{\songti \zihao{2}\bfseries 液晶光电效应综合实验}
\titleformat*{\section}{\songti\zihao{4}\bfseries}
\titleformat*{\subsection}{\songti\zihao{5}\bfseries}
\renewcommand\thesection{\arabic{section}}
\author{王启骅 PB20020580}
\begin{document}
由于水槽两边为绝缘边界,可认为是理想情况下极板是铝箔平行板电容器的匀强电场,则有电势分布
\begin{equation}
	U(x)=U_0(1-\frac{x}{s})\nonumber
\end{equation}


圆柱形容器半径为R,电极半径为a。圆柱形容器周围包裹铝箔,并接电源负极,电极连接电源正极,电压为$ U_0 $


双电层分为吸附层(Stern),扩散层(Guoy) 。扩散层正负离子分布满足玻尔兹曼分布
\begin{equation}
	n_+=n_{0+}exp[-\frac{ze\phi}{kT}] \nonumber
\end{equation}
\begin{equation}
	n_-=n_{0-}exp[\frac{ze\phi}{kT}] \nonumber
\end{equation}


由麦克斯韦方程
\begin{equation}
	\bigtriangledown^2\phi=-\frac{\rho}{\epsilon} \nonumber
\end{equation}


联立并得到
\begin{equation}
	\dfrac{d^2\phi}{dr^2}=\frac{2n_0ze}{\epsilon}sinh(\frac{ze\phi}{kT})
	\nonumber
\end{equation}


由$ \phi $为低电势近似得到
\begin{equation}
	\dfrac{d^2\phi}{dr^2}=\frac{2n_0z^2e^2}{\epsilon kT}\phi
	\nonumber
\end{equation}


解得双电层产生的电势偏移量
\begin{equation}
	\phi=\Phi_sexp[-\sqrt{\frac{2n_0z^2e^2}{\epsilon kT}}r]
	\nonumber
\end{equation}



\end{document}